\chapter{The Mean}

The simplest example for least squares, the mean.

\section{Linear System}  %    S    S    S    S    S    S    S    S    S
Given a sequence of numbers $\lst{x_{k}}_{k=1}^{m}$, which number $\mu$ best characterizes the sequence?
  \begin{equation*}   %  =   =   =   =   =
     \begin{array}{cccc}
         \A{} & \mu & = & T \\
         \ones{} & \mu & = & \Tbev
    %\label{eq:}
    \end{array}
  \end{equation*}
\section{Data}  %    SS   SS   SS   SS   SS   SS   SS   SS   SS

\section{Solution Methods}  %    SS   SS   SS   SS   SS   SS   SS   SS   SS

\subsection{Solution via Linear Algebra}  %    SSS   SSS   SSS   SSS   SSS   SSS   SSS   SSS   SSS
\subsection{Solution vis Calculus}  %    SSS   SSS   SSS   SSS   SSS   SSS   SSS   SSS   SSS
%
\begin{equation}
    M(\mu) = \sum_{k}^{m} \paren{ x_{k} - \mu }^2
\end{equation}
%
Shorthand
  \begin{equation}   %  =   =   =   =   =
  %\begin{split}
    \pd{M(\mu)}{\mu} \equiv \partial_{\mu} M
    %\label{eq:}
  %\end{split}
  \end{equation}
%
\begin{equation}
    \partial_{\mu} M = 0
\end{equation}
%

%
\begin{equation}
    -2 \paren{ x_{k} - \mu } = 0
\end{equation}
%
%
\begin{equation}
    \mu = \frac{\sum_{k=1}^{2}} {m}
\end{equation}
%
\begin{figure}[htbp]
  \begin{center}
      \includegraphics[ width = 4in ]{\pathgraphics tricks/mean/"mean merit"}
    \caption{default}
    \label{default}
  \end{center}
\end{figure}

\endinput   %   =   =   =   =   =   =   =   =   =   =   =   =   =   =   =   =   =   =   =   =   =   =   =   =   =   =   =   =   =   =   =   =   =   =   =   =