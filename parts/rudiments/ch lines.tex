\chapter{Playing With Lines}

The method of least squares is a wonderfully versatile tool, and we will get a sense of this from looking at the canonical equation for a line
  \begin{equation*}   %  =   =   =   =   =
   %\begin{split}
      y(x) = a_{0} + a_{1} x.
   %\end{split}
   %\label{eq:}
  \end{equation*}
There are two ways to exploit the equation. One way is to input a set of measurements $\lst{x_{k},y_{k}}_{k=1}^{m}$, and find the best parameters $\paren{a_{0},a_{1}}$, in the least squares sense. 

The focus on this chapter is to start with a set of parameters $\paren{a_{0},a_{1}}$ and to find the best set $\paren{x, y}$, in the least squares sense.	

\begin{table}[htbp]  % + + + + T A B L E
    \caption{Given the parameters for a line find the solution locus.}
    		\begin{center}
        	\begin{tabular}{lcl}
            %
            input & $\rightarrow$ & output \\\hline
            %
            parameters $a_{0}$, $a_{1}$ && set $(x, y)$\\
      			%
						(intercept, slope) && (least squares solution)	
            %
        	\end{tabular}
    		\end{center}
    %\label{default}
\end{table}%

\section{A Single Line}  %    S    S    S    S    S    S    S    S    S

Given the parameters $a=\mat{c}{a_{0}\\a_{1}}$, find the set $\mat{c}{x\\y}$ which comprises the least squares solution for
  \begin{equation}   %  =   =   =   =   =
   %\begin{split}
      y = a_{0} + a_{1} x, \qquad a_{0} \ne 0.
   %\end{split}
   \label{eq:myline}
  \end{equation}

\begin{figure}[htbp] %  figure placement: here, top, bottom, or page
   \centering
   \begin{overpic}[ scale = \myscale ]
	   {\pathgraphics tricks/one_line/"line- y = mx + b"}
        %
    	\put(50,-3) {$x$}
    	\put(-4,31) {$y$}
	    %
   \end{overpic}
   \caption{A sample line.}
   %\label{fig:three lines oval}
\end{figure}

We can recast \eqref{eq:myline} in this form
  \begin{equation*}   %  =   =   =   =   =
   %\begin{split}
      -a_{1} x + y = a_{0},
   %\end{split}
   %\label{eq:}
  \end{equation*}
which reveals the linear system
  \begin{equation}   %  =   =   =   =   =
  %\begin{split}
    \mat{rc}{ -a_{1} & 1 } \mat{c}{x\\y} = \mat{c}{a_{0}}.
    \label{eq:one line problem}
  %\end{split}
  \end{equation}
The least squares solution for this problem is set $x_{LS}$ defined as
  \begin{equation*}   %  =   =   =   =   =
   %\begin{split}
      x_{LS} = \lst{x\in\real{2}\colon \norms{\A{}x - a_{0}} \text{is minimized}} .
   %\end{split}
   %\label{eq:}
  \end{equation*}
The target of minimization, the merit function, is written so:
  \begin{equation}   %  =   =   =   =   =
   %\begin{split}
      M\paren{x,y} = \paren{ - a_{1} x + y - a_{0}}^{2}.
   %\end{split}
   \label{eq:one line merit}
  \end{equation}

The problem statement is complete and summarized in table \ref{tab:one line inputs}. We can't stress enough the importance of a clear and succinct problem statement. 
  
\mg{Certainly, the heavy artillery of the SVD could be brought to bear. Yet simple geometric reasoning will produce a solution and reinforce basic insights. Afterwards, the SVD will become a trivial exercise.

Equation \eqref{eq:myline} describes a line, a continuous set of points. Referring back to the boxed least squares solution in \eqref{eq:general soln}, the least squares solution may be the 

Which point on the line is closet to the origin? One way to solve for this is use the fact that the shortest line through the origin which connects with \eqref{eq:myline} is perpendicular to \eqref{eq:myline}. This is the orthogonal projection theorem.}

  \begin{table}[h]  %  T A B L E
    \caption{Problem statement for least squares solution for a single line.}
    \begin{center}
      \begin{tabular}{lll}
        %
        \bf{trial function} & $y(x) = a_{0} + a_{1} x$ \\
        \bf{residual error} & $r(x) = y - a_{0} - a_{1} x$ \\
        \bf{merit function} & $M(a) = \paren{y - a_{0} - a_{1} x}^{2}$\\
        \bf{inputs}         & $a_{0}$ & $y-$axis intercept \\
                            & $a_{1}$ & slope \\
        \bf{results}        & $\mat{c}{x\\y}_{LS} = \bl{\mat{c}{x\\y}_{part}} + \rd{\mat{c}{x\\y}_{homog}}$   \\
        \bf{\# of parameter sets} & $m = 1$ & rows in $\A{}$ \\
        \bf{\# of parameters}   & $n = 2$ & columns in $\A{}$ \\
        \bf{system matrix}  & $\A{} \in \real{1 \times 2}_{1}$ \\
        \bf{linear system}  & $\mat{cc}{-a_{1} & 1} 
                               \mat{c}{x \\ y} = 
                               \mat{c}{a_{0}}$ \\
        %\bf{ideal solution} & NA \\
        \bf{input data}     & $a_{0}$, $a_{1}$
        %
      \end{tabular}
    \end{center}
  \label{tab:one line inputs}
  \end{table}%

\begin{figure}[htbp] %  figure placement: here, top, bottom, or page
   \centering
   \begin{overpic}[ scale = \myscale ]
	   {\pathgraphics tricks/one_line/fan}
        %
    	\put(50,-3) {$x$}
    	\put(-4,31) {$y$}
	    %
   \end{overpic}
   \caption{Every point on the line satisfies the least squares criterion; the particular solution is the vector of minimum length.}
   \label{fig:one line fan}
\end{figure}

\subsection{Geometric Solution}  %    SS    SS    SS    SS    SS    SS    SS    SS    SS
Every point on the line \eqref{eq:myline} is a point of 0 residual, therefore every point on the line is a solution point. How does the method of least squares select a special point as the solution?

Figure \ref{fig:one line fan} shows a sampling of points on the line. The critical insight is that each of the arrows represents a solution vector. The particular solution, the least squares solution, is the vector of \emph{minimum length}. By the orthogonal projection theorem, the point on the line closest to the origin is the orthogonal projection of the line onto the origin. Center a circle at the origin and increase the radius until the curve first touches the line at one point. We now have an osculating circle which kisses the line at the point closest to the origin.

Figure \ref{fig:one line osculate} shows the osculating circle and the full solution to the least squares problem. The particular solution is the blue dot; the homogenous solution the red, dashed line. The solution vector, in black, is orthogonal to the solution line.
\begin{figure}[htbp] %  figure placement: here, top, bottom, or page
   \centering
   \begin{overpic}[ scale = \myscale ]
	   {\pathgraphics tricks/one_line/"line- circle, dot line, bracket"}
        %
    	\put(50,-3) {$x$}
    	\put(-4,31) {$y$}
	    %
   \end{overpic}
   \caption{The point on the line closest to the origin.}
   \label{fig:one line osculate}
\end{figure}

Solving for the blue dot is an exercise in algebra. Let $y\perp$ be the line containing the origin and the solution point, and let $y(x)$ be the solution in \eqref{eq:myline}. The intersection point solves the equation
% = =  e q u a t i o n
  \begin{equation}
     \begin{split}
       %
         y_{\perp} &= y(x) \\
       %
        &\Downarrow \\
       %
         -\frac{1}{a_{1}} x &= a_{0} + a_{1}x.
       %
       % 
     \end{split}
    \label{eq:one line osculating}
  \end{equation}
% = =
We now have the particular solution for the least squares problem:
  \begin{equation*}   %  =   =   =   =   =
   %\begin{split}
      \bl{\mat{c}{x\\y}_{p}} = \frac{a_{0}} {1 + a_{1}^{2}} \bl{\mat{c}{-a_{1}\\1}}
   %\end{split}
   %\label{eq:}
  \end{equation*}
The homogenous solution represents the other points on the thick line.

Stepping back, the parametric representation of any line can be written by 
  \begin{equation*}   %  =   =   =   =   =
   %\begin{split}
      p(\tau) = p_{0} + \tau \mat{c}{1\\a_{1}}
   %\end{split}
   %\label{eq:}
  \end{equation*}
where $p_{0}$ is an arbitrary point on said line and $\tau\in\real{}$. For the problem of interest, $p_{0}$ is the particular solution, and the full least squares solution becomes
  \begin{equation}   %  =   =   =   =   =
   %\begin{split}
      \mat{c}{x\\y}_{LS} = \frac{a_{0}} {1 + a_{1}^{2}} 
        \underbrace{\bl{\mat{c}{-a_{1}\\1}}}_{particular} +
        \underbrace{\tau \rd{\mat{c}{1\\a_{1}}}}_{homogeneous}  .
   %\end{split}
    \label{eq:one line full soln}
  \end{equation}
The range space decomposition for the solution is displayed in figure \ref{fig:one line resolved}.
\begin{figure}[htbp] %  figure placement: here, top, bottom, or page
   \centering
   \begin{overpic}[ scale = \myscale ]
	   {\pathgraphics tricks/one_line/"line- blue dot, red line"}
        %
        \put(21,42) {\rd{$x_{h}$}}
        \put(27,48) {\rd{$\pna z$}}
        %
        \put(55,25) {\colorbox{white}{\bl{$x_{p}$}}}
        \put(61,31) {\bl{$\Ap a_{0}$}}
        %
    	\put(50,-3) {$x$}
    	\put(-4,31) {$y$}
	    %
   \end{overpic}
   \caption{The least squares solution for \eqref{eq:one line problem} resolved into range space (blue) and null space components (red).}
   \label{fig:one line resolved}
\end{figure}

The solution for the problem in table \ref{tab:one line inputs} with specific values for $a_{0}$ and $a_{1}$ is summarized in table \ref{tab:one line solution}. These are the parameters used to represent the plots within this section.

  \begin{table}[t]  %  T A B L E
    \caption{Results for best line with $a_{0} = \frac{1}{2}$, $a_{1} = -\frac{1}{2}$.}
    \begin{center}
      \begin{tabular}{lll}
        %
        \bf{input parameters} & $a_{0} =  \frac{1}{2}$ & intercept \\
                              & $a_{1} = -\frac{1}{2}$ & slope \\
        \bf{computed solution} & $\mat{c}{x\\y}_{LS} = \bl{\frac{1}{5}\mat{c}{1\\2}} + \tau \rd{\mat{c}{1\\-1/2}}$ & $\tau \in \cmplx{}$\\
        \bf{solution error} & 0 & exact solution \\
        %\bf{ideal solution} & $\mat{c}{ \tilde{a}_{0} \\ \tilde{a}_{1} } = \mat{c}{0\\10}$ \\
        \bf{$\rtr{*}$} & $0$ \\
        \bf{curvature matrix $\wxi{*}$} & $\frac{1}{4}\mat{rr}{1 & \mg{2} \\ \mg{2} & 4 }$\\[5pt]
        \bf{problem statement} & table \ref{tab:one line inputs} \\
        %\bf{input data}        & table \ref{tab:bevington data and results} \\
        \bf{plots}             & figure \ref{fig:one line resolved} & 1. solution\\
           & figure \ref{fig:one lines merit} & 2. merit function \\
        %
      \end{tabular}
    \end{center}
  \label{tab:one line solution}
  \end{table}%

\subsection{Solution Space}  %    SS    SS    SS    SS    SS    SS    SS    SS    SS
It's powerful to see the actual minimization that least squares delivers. To do so, we present a contour plot in figure \ref{fig:one lines merit} showing the level surfaces of the merit function in the space of the solution parameters. The plot confirms the results in \eqref{eq:one line full soln}. 

\begin{figure}[htbp] %  figure placement: here, top, bottom, or page
   \centering
   \begin{overpic}[ scale = \myscale ]
	   {\pathgraphics tricks/one_line/"one line merit"}
	    %
	    \put(25,58) {$M(a_{0},a_{1}) = \paren{y - a_{0} - a_{1}x}^{2}$}
        %
    	\put(50,-3) {$x$}
    	\put(-4,27) {$y$}
	    %
   \end{overpic}
   \caption{A contour plot of the merit function showing the particular solution (blue dot) and homogeneous solution (red dashes).}
   \label{fig:one lines merit}
\end{figure}

The minimum value of 0 is attained along a line, not just at a point. This red, dashed line represents the null space component, the homogenous solution. The blue dot represents the range space component, the particular solution. 

\subsection{SVD}  %    SS    SS    SS    SS    SS    SS    SS    SS    SS
We can construct the \asvd \ by inspection. Due to the simplicity of this problem, the domain matrices can be constructed in any order.

\subsubsection{$\U{}$: codomain matrix}  %  SSS   SSS   SSS   SSS   SSS   SSS   SSS   SSS   SSS
The target matrix $\A{} = \mat{cc}{-a_{1} & 1}$ has $m=1$ row, which implies the $\cmplx{\byy{1}}$ matrix for the codomain is
  \begin{equation*}   %  =   =   =   =   =
   %\begin{split}
      \U{} = \mat{c}{1}.
   %\end{split}
   %\label{eq:}
  \end{equation*}

\subsubsection{$\V{}$: domain matrix}  %   SSS   SSS   SSS   SSS   SSS   SSS   SSS   SSS   SSS
With only one vector for the row space, it must describe the range space. The orthogonal compliment constitutes the null space. As the row vectors have dimension $n=2$, the $\cmplx{\byy{2}}$ matrix for the domain is then
% = =  e q u a t i o n
  \begin{equation}
    %\begin{split}
      \V{} = \cvblockf{} = \frac{1}{\onelen} \mat{cc}{ \bl{-a_{1}} & \rd{1} \\ \bl{1} & \rd{a_{1}} } .
    %\end{split}
    %\label{eqn:}
  \end{equation}
% = =

\subsubsection{$\ess{}$: matrix of singular values	}  %   SSS   SSS   SSS   SSS   SSS   SSS   SSS   SSS   SSS
Because the linear system is rank $\rho=1$, there will be only one singular value $\sigma_{1}$ and the $\real{\byy{1}}$ matrix of singular values is
  \begin{equation*}   %  =   =   =   =   =
  %\begin{split}
    \ess{} = \mat{c}{\sigma_{1}}
    %\label{eq:}
  %\end{split}
  \end{equation*}
With the sabot padding of zeros, 
the matrix $\sig{} = \mat{cc}{\sigma_{1} & 0}$. Of course one could compute the lone eigenvalue of the product matrix $\wx{*}$, but the inspection method is faster. What value of $\sigma_{1}$ satisfies the following equation
  \begin{equation*}   %  =   =   =   =   =
    \begin{array}{ccccc}
      %
      \A{} &=& \U{} & \Sigma & \V{*} \\
      %
      \mat{cc}{-a_{1} & 1} &=& 
      \mat{c}{1} & 
      \mat{cc}{\sigma_{1} & 0} & 
      \frac{1}{\onelensq}
      \mat{cc}{\bl{-a_{1}} & \bl{1} \\ \rd{1} & \rd{a_{1}} } ?
      %
    \end{array}
   %\label{eq:}
  \end{equation*}

\subsubsection{Assemble components}  %   SSS   SSS   SSS   SSS   SSS   SSS   SSS   SSS   SSS
The complete \asvd \ is
  \begin{equation*}   %  =   =   =   =   =
    \begin{array}{ccccc}
      %
      \A{} &=& \U{} & \Sigma & \V{*} \\
      %
      \mat{cc}{-a_{1} & 1} &=& 
      \mat{c}{1} & 
      \mat{cc}{\sqrt{\onelen} & 0} & 
      \frac{1}{\onelensq}
      \mat{cc}{\bl{-a_{1}} & \bl{1} \\ \rd{1} & \rd{a_{1}} } .
      %
    \end{array}
   %\label{eq:}
  \end{equation*}

\subsubsection{Construct pseudoinverse}  %   SSS   SSS   SSS   SSS   SSS   SSS   SSS   SSS   SSS
The pseudoinverse matrix is
  \begin{equation*}   %  =   =   =   =   =
   %\begin{split}
      \Ap = \V{} \sig{\sym} \U{*} = \frac{1} {\onelen} \mat{c}{-a_{1} \\ 1}.
   %\end{split}
   %\label{eq:}
  \end{equation*}
Thus, we have computed the pseudoinverse for a vector, in agreement with Laub's results \cite{Laub2005}[]. Given a vector $w\icm$:
  \begin{equation}   %  =   =   =   =   =
  %\begin{split}
    w^{\sym} = \frac{w^{*}} {\norm{w}}
    \label{eq:vectorpi}
  %\end{split}
  \end{equation}

\subsubsection{Solve the least squares problem}  %   SSS   SSS   SSS   SSS   SSS   SSS   SSS   SSS   SSS
Using this pseudoinverse to find the particular solution $\bl{x_{p}}$ for the least squares problem leaves
  \begin{equation}   %  =   =   =   =   =
      \bl{x_{p}} = \bl{\A{\dagger} a_{0}} = \frac{a_{0}}{\onelen} \bl{\mat{c}{-a_{1} \\ 1}},
   \label{eq:one line soln}
  \end{equation}
in agreement with \eqref{eq:one line full soln}. The full least squares solution includes the null space term for the homogenous solution $\rd{x_{h}}$:
  \begin{equation}   %  =   =   =   =   =
   \begin{array}{ccccccl}
     x_{LS} 
     	 &=& \frac{a_{0}}{\onelen} & \bl{\mat{c}{-a_{1} \\ 1}} & + & \alpha \rd{\mat{c}{1 \\ a_{1}}}, & \alpha \ic \\
     	 &=& &\bl{x_{p}} & + & \rd{x_{h}}
    %\label{eq:}
   \end{array}
  \end{equation}

\section{Two Lines}  %    S    S    S    S    S    S    S    S    S
The game is a bit more interesting when we consider two lines; there is now the classic $\paren{0,1,\infty}$ hierarchy of solutions:
  \begin{table}[htbp]  %  T A B L E
    \caption{Existence and uniqueness with two lines.}
    \begin{center}
      \begin{tabular}{clcc}
        %
        number of solutions & placement of lines & existence & uniqueness \\\hline
        %
        0 & parallel, not coincident  & no & yes \\
        %
        1 & crossing & yes & yes \\
        %
        $\infty$ & coincident & yes & no
        %
      \end{tabular}
    \end{center}
  %\label{tab:?}
  \end{table}%

The exploration begins with the simplest case: intersecting lines.

\subsection{Crossing Lines}  %    SS    SS    SS    SS    SS    SS    SS    SS    SS
Our first exposure to the algebra of linear systems involves simultaneous ratification of two criteria. These criteria are expressed as two lines,
  \begin{equation*}   %  =   =   =   =   =
    \begin{split}
      y_{1} &= m_{1}x + b_{1} \\
      y_{2} &= m_{2}x + b_{2}
    %\label{eq:}
    \end{split}
  \end{equation*}
and the demand is both be true simultaneously.

Elementary means produces the solution
  \begin{equation}   %  =   =   =   =   =
  %\begin{split}
    x_{p} = \frac{b_{2} - b_{1}} {m_{1} - m_{2}}.
    \label{eq:crossing}
  %\end{split}
  \end{equation}
The denominator signals distress when the lines are parallel, that is when $m_{1} = m_{2}$. What happens when the difference in the slope of the lines is small?

\subsubsection{Troubling Sequences}  %    SSS    SSS    SSS    SSS    SSS    SSS    SSS    SSS    SSS
Let $k\in\mathbb{N}$ and add a perturbation by letting $m_{2} = m_{1} + 2^{-k}$? This is the case where $m_{2}$ approaches $m_{1}$ from above. What happens in the limit $k\to \infty?$
  \begin{equation*}   %  =   =   =   =   =
  %\begin{split}
    \lim_{m_{2}\to m_{1}^{+}} x_{p} = \frac{b_{2} - b_{1}} {2^{-k}} \to \infty
    %\label{eq:}
  %\end{split}
  \end{equation*}
But what happens when $m_{2}$ approaches $m_{1}$ from below? 
  \begin{equation*}   %  =   =   =   =   =
  %\begin{split}
    \lim_{m_{2}\to m_{1}^{-}} x_{p} = \frac{b_{2} - b_{1}} {-2^{-k}}  \to -\infty
    %\label{eq:}
  %\end{split}
  \end{equation*}
  
These machinations demonstrate that \eqref{eq:crossing} is not suitable for computing a solution when the lines are parallel. The method of least squares provides a solution.

\subsection{Parallel Lines}  %    SS    SS    SS    SS    SS    SS    SS    SS    SS
For parallel lines there is but a single slope parameter $m$:
  \begin{equation*}   %  =   =   =   =   =
    \begin{split}
      y_{1} &= mx + b_{1} \\
      y_{2} &= mx + b_{2},
    %\label{eq:}
    \end{split}
  \end{equation*}
which is equivalent to the following linear system:
  \begin{equation*}   %  =   =   =   =   =
  %\begin{split}
    \mat{rc}{-m & 1 \\ -m & 1} \mat{c}{x\\y} = \mat{c}{b_{1} \\ b_{2}}.
    %\label{eq:}
  %\end{split}
  \end{equation*}
The task is to compute the \asvd \ of the matrix.
  \begin{equation*}   %  =   =   =   =   =
  %\begin{split}
    \A{} = \mat{rc}{-m & 1 \\ -m & 1}
    %\label{eq:}
  %\end{split}
  \end{equation*}
%
  \begin{equation*}   %  =   =   =   =   =
  %\begin{split}
    \A{*}\A{} = 2 \mat{rr}{m^{2} & -m \\ -m & 1}
    %\label{eq:}
  %\end{split}
  \end{equation*}
%
  \begin{equation*}   %  =   =   =   =   =
  %\begin{split}
    \lambda \paren{\A{*}\A{}} = \paren{2\paren{1+m^{2}},0}
    %\label{eq:}
  %\end{split}
  \end{equation*}
%
%
  \begin{equation*}   %  =   =   =   =   =
  %\begin{split}
  	\sigma = \paren{\sqrt{2\paren{1+m^{2}}}}
      %\label{eq:}
  %\end{split}
  \end{equation*}
The \asvd \ is
%
  \begin{equation*}   %  =   =   =   =   =
    \begin{array}{ccccc}
      \A{} & = & \U{} & \Sigma & \V{*} \\
        & = & \frac{1} {\sqrt{2}} \mat{cr}{1 & -1 \\ 1 & 1} 
            & \mat{cc}{\sqrt{2(1 + m^{2})} & 0}
            & \frac{1} {\sqrt{1 + m^{2}}} \mat{rc}{-m & 1 \\ 1 & m}
    %\label{eq:}
    \end{array}
  \end{equation*}
Where Do Parallel Lines Cross? The provocative question which opens this section has an obvious answer in Euclidean space. There is no such point; parallel lines never cross. Yet, if we input these lines as a linear system, we compute a least squares solution. What is the significance of the least squares solution?

Consider this tantalizing example. The two lines,
  \begin{equation*}   %  =   =   =   =   =
    \begin{split}
      y(x) &= \half - \half x, \\
      y(x) &= 1 - \half x,
    \end{split}
   %\label{eq:}
  \end{equation*}
are plotted in figure \ref{fig:tantalizing} along with the least squares solution is 
  \begin{equation*}   %  =   =   =   =   =
   %\begin{split}
      x_{LS} = \Ap b = \frac{1}{10}\mat{c}{ 3 \\ 6}.
   %\end{split}
   %\label{eq:}
  \end{equation*}
What is so special about this point?
\begin{figure}[htbp] %  figure placement: here, top, bottom, or page
   \centering
     \includegraphics[ width = 3in ]{\pathgraphics "tricks"/two_lines/"parallel point"} 
   \caption{Parallel lines and the least squares solution.}
   \label{fig:tantalizing}
\end{figure}

\subsection{Intersecting Lines}
  \begin{equation*}   %  =   =   =   =   =
   \begin{split}
      y &= 1, \\
      y &= x .
   \end{split}
   %\label{eq:}
  \end{equation*}

  \begin{equation*}   %  =   =   =   =   =
   %\begin{split}
      \mat{rr}{0 & 1 \\ -1 & 1 }\mat{c}{x\\y} = \mat{c}{1\\0}
   %\end{split}
   %\label{eq:}
  \end{equation*}

\subsection{Overlapping Lines}  %    SS    SS    SS    SS    SS    SS    SS    SS    SS

  \begin{equation*}   %  =   =   =   =   =
    \begin{split}
      y_{1} &= mx + b \\
      y_{2} &= mx + b
    %\label{eq:}
    \end{split}
  \end{equation*}

  \begin{equation*}   %  =   =   =   =   =
  %\begin{split}
    \mat{rc}{-m & 1 \\ -m & 1} \mat{c}{x\\y} = \mat{c}{b \\ b}
    %\label{eq:}
  %\end{split}
  \end{equation*}


\section{Three Lines}  %    S    S    S    S    S    S    S    S    S
Start with three distinct lines. The first line represents a constant value. The second line has a fixed slope, the third has a variable slope. We will study how the solution varies as this slope $m$ is varied.
  \begin{equation*}   %  =   =   =   =   =
     \begin{split}
       %
       y_{1}(x) &= 1, \\
       y_{2}(x) &= 1 - x, \\
       y_{3}(x) &= m x.
       %
     \end{split}
   %\label{eq:}
  \end{equation*}

%  \begin{equation*}   %  =   =   =   =   =
%    \begin{split}
%      &y = 1 \\
%      x + &y = 1 \\
%      -mx +&y = 0
%    \end{split}
%   %\label{eq:}
%  \end{equation*}

  \begin{table}[htbp]
  \caption{Rewriting the equations $y=mx+b$ as a linear system.}
  \begin{center}
    \begin{tabular}{lclcccrcrcl}
      %
      $y_{1}(x)$ &=&  1       && $\Rightarrow$ && &&          $y$ &=& 1 \\
      %
      $y_{2}(x)$ &=&  $1 - x$ && $\Rightarrow$ && $x$   & + & $y$ &=& 1 \\
      %
      $y_{3}(x)$ &=&  $m x$   && $\Rightarrow$ && $-mx$ & + & $y$ &=& 0 \\
      %
    \end{tabular}
  \end{center}
  %\label{default}
  \end{table}%


  \begin{equation*}   %  =   =   =   =   =
   %\begin{split}
      \mat{rc}{0 & 1 \\ 1 & 1 \\ -m & 1 } 
      \mat{c}{x \\ y} = 
      \mat{c}{1\\1\\0}
   %\end{split}
   %\label{eq:}
  \end{equation*}

Merit function
  \begin{equation*}   %  =   =   =   =   =
   %\begin{split}
      M(x,y) = \normts{\A{}\mat{c}{x\\y}-b}
   %\end{split}
   %\label{eq:}
  \end{equation*}

  \begin{equation*}   %  =   =   =   =   =
   %\begin{split}
      \A{*} \cdot \A{} = 
      \mat{lc}{1+m^{2} & 1-m \\ 1-m & 3}
   %\end{split}
   %\label{eq:}
  \end{equation*}

  \begin{equation*}   %  =   =   =   =   =
    \begin{split}
      \tr {\A{}} &= 4 + m^{2} \\
      \det \paren{\A{}} &= 2 + 2m + 2m^{2}
    \end{split}
   %\label{eq:}
  \end{equation*}

  \begin{equation*}   %  =   =   =   =   =
   %\begin{split}
      p\paren{\lambda} = \lambda^{2} - \lambda \, \tr{\W{}} + \det\paren{\W{}}
   %\end{split}
   %\label{eq:}
  \end{equation*}
  \begin{equation*}   %  =   =   =   =   =
   %\begin{split}
      p\paren{\lambda} = 0
   %\end{split}
   %\label{eq:}
  \end{equation*}

  \begin{equation*}   %  =   =   =   =   =
   %\begin{split}
      \lambda_{\pm} = \frac{\tr{\W{}} \pm \sqrt{ \tr{\W{}}^{2} - 4 \det\paren{\W{}} } } {2}
   %\end{split}
   %\label{eq:}
  \end{equation*}


  \begin{equation*}   %  =   =   =   =   =
   %\begin{split}
      \sigma = 2^{-1/2} \sqrt{m^{2} + 4 \pm \factor}
   %\end{split}
   %\label{eq:}
  \end{equation*}

  \begin{equation*}   %  =   =   =   =   =
      \xi = \factor
   %\label{eq:}
  \end{equation*}

%\begin{table}[htbp]
%\caption{Least squares solution for three distinct lines as the parameter $m$ varies from 0 to $\infty$.}
%    \begin{center}
%        \begin{tabular}{ccc}
%           %
%           & Domain: Graphs & Domain: Merit Function \\\hline
%           %
%           \raisebox{1.5\height}{$m=0$} &
%           \includegraphics[ width = 2in ]{\pathgraphics tricks/three_lines/"three lines m = 0"} &
%           \includegraphics[ width = 2.1in ]{\pathgraphics tricks/three_lines/"merit function m = 0"} \\[10pt]
%           %
%           \raisebox{-2.5\height}{$m=1$} &
%           \includegraphics[ width = 2in ]{\pathgraphics tricks/three_lines/"three lines m = 1"} &
%           \includegraphics[ width = 2.1in ]{\pathgraphics tricks/three_lines/"merit function m = 1"} \\[10pt]
%           %
%           $m=2$ &
%           \includegraphics[ width = 2in ]{\pathgraphics tricks/three_lines/"three lines m = 2"} &
%           \includegraphics[ width = 2.1in ]{\pathgraphics tricks/three_lines/"merit function m = 2"} \\[10pt]
%           %
%           $m=5$ &
%           \includegraphics[ width = 2in ]{\pathgraphics tricks/three_lines/"three lines m = 5"} &
%           \includegraphics[ width = 2.1in ]{\pathgraphics tricks/three_lines/"merit function m = 5"} \\[10pt]
%           %
%           $m=\infty$ &
%           \includegraphics[ width = 2in ]{\pathgraphics tricks/three_lines/"three lines m = inf"} &
%           \includegraphics[ width = 2.1in ]{\pathgraphics tricks/three_lines/"merit function m = inf"} \\
%           %
%        \end{tabular}
%    \end{center}
%\label{tab:three lines}
%\end{table}%

\break
\begin{table}[htbp]
\caption{Least squares solution for three distinct lines as the parameter $m$ varies from 0 to $\infty$.}
    \begin{center}
        \begin{tabular}{lc}
           %
           $m$ & Domain: Graphs \\\hline
           %
           \raisebox{8\height}{$m=0$} &
           \includegraphics[ width = 2.5in ]{\pathgraphics tricks/three_lines/"three lines m = 0"} \\[7pt]
           %
           \raisebox{8\height}{$m=1$} &
           \includegraphics[ width = 2.5in ]{\pathgraphics tricks/three_lines/"three lines m = 1"}\\[7pt]
           %
           \raisebox{8\height}{$m=5$} &
           \includegraphics[ width = 2.5in ]{\pathgraphics tricks/three_lines/"three lines m = 5"}\\[7pt]
           %
           \raisebox{12\height}{$m=\infty$} &
           \includegraphics[ width = 2.5in ]{\pathgraphics tricks/three_lines/"three lines m = inf"}
           %
        \end{tabular}
    \end{center}
\label{tab:three lines graphs}
\end{table}%

\begin{table}[htbp]
\caption{Least squares solution for three distinct lines as the parameter $m$ varies from 0 to $\infty$.}
    \begin{center}
        \begin{tabular}{lc}
           %
           $m$ & Domain: Merit function \\\hline
           %
           \raisebox{8\height}{$m=0$} &
           \includegraphics[ width = 2.5in ]{\pathgraphics tricks/three_lines/"merit function m = 0"} \\[7pt]
           %
           \raisebox{8\height}{$m=1$} &
           \includegraphics[ width = 2.5in ]{\pathgraphics tricks/three_lines/"merit function m = 1"}\\[7pt]
           %
           \raisebox{8\height}{$m=5$} &
           \includegraphics[ width = 2.5in ]{\pathgraphics tricks/three_lines/"merit function m = 5"}\\[7pt]
           %
           \raisebox{12\height}{$m=\infty$} &
           \includegraphics[ width = 2.5in ]{\pathgraphics tricks/three_lines/"merit function m = inf"}
           %
        \end{tabular}
    \end{center}
\label{tab:three lines merit}
\end{table}%

%\begin{landscape}
%\begin{table}[htbp]
%\caption{Least squares solution for three distinct lines as the parameter $m$ varies from 0 to $\infty$.}
%    \begin{center}
%        \begin{tabular}{ccc}
%           %
%           & Domain: Graphs & Domain: Merit Function \\\hline
%           %
%           \raisebox{10\height}{$m=0$} &
%           \includegraphics[ width = 2.5in ]{\pathgraphics tricks/three_lines/"three lines m = 0"} &
%           \includegraphics[ width = 2.675in ]{\pathgraphics tricks/three_lines/"merit function m = 0"} \\[10pt]
%           %
%           \raisebox{8\height}{$m=1$} &
%           \includegraphics[ width = 2.5in ]{\pathgraphics tricks/three_lines/"three lines m = 1"} &
%           \includegraphics[ width = 2.675in ]{\pathgraphics tricks/three_lines/"merit function m = 1"} \\[10pt]
%           %
%           \raisebox{7\height}{$m=5$} &
%           \includegraphics[ width = 2.5in ]{\pathgraphics tricks/three_lines/"three lines m = 5"} &
%           \includegraphics[ width = 2.675in ]{\pathgraphics tricks/three_lines/"merit function m = 5"} \\[10pt]
%           %
%           \raisebox{5\height}{$m=\infty$} &
%           \includegraphics[ width = 2.5in ]{\pathgraphics tricks/three_lines/"three lines m = inf"} &
%           \includegraphics[ width = 2.675in ]{\pathgraphics tricks/three_lines/"merit function m = inf"} \\
%           %
%        \end{tabular}
%    \end{center}
%\label{tab:three lines}
%\end{table}%
%\end{landscape}

  \begin{equation}   %  =   =   =   =   =
  %\begin{split}
    p(m) = \frac{1} {2\paren{m^{2} + m + 1}} \mat{c}{ b_{1}\paren{m-1} + b_{2}\paren{m+2} + b_{3} \paren{-2m-1} \\
                                                      b_{1}\paren{m^{2}+1} + b_{2} m\paren{m+1} + b_{3} \paren{m+1} }
    \label{eq:p(m)}
  %\end{split}
  \end{equation}

%\begin{figure}[htbp] %  figure placement: here, top, bottom, or page
%   \centering
%   \includegraphics[ width = 3.5in ]{\pathgraphics tricks/three_lines/"three lines oval"} 
%   \caption{example caption}
%   \label{fig:three lines oval}
%\end{figure}

\begin{figure}[htbp] %  figure placement: here, top, bottom, or page
   \centering
   \begin{overpic}[ scale = \myscale ]
	   {\pathgraphics tricks/three_lines/"three lines oval"}
        %
%        \put(45,48) {$m\rightarrow\infty^{-}$}
%        \put(61,40) {$m\rightarrow\infty^{+}$}
		\put(55,43) {$m=\infty$}
        %
        \put(17,43) {$m=-1$}
        %
        \put(76.5,17.5) {$m=1$}
        %
        \put(83,8) {$m=0$}
        %
    	\put(51,-3) {$x$}
    	\put(-4,31) {$y$}
	    %
   \end{overpic}
   \caption{Trajectory of the solution point $p(m)$ in \eqref{eq:p(m)} for $-\infty < m < \infty$.}
   \label{fig:three lines oval}
\end{figure}

\endinput