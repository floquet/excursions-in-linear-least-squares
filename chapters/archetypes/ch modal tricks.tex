\chapter{Tricks}

\section{Where Do Parallel Lines Cross?}  %    S    S    S    S    S    S    S    S    S
The provocative question which opens this section has an obvious answer in Euclidean space. There is no such point; parallel lines never cross. Yet, if we input these lines as a linear system, we compute a least squares solution. What is the significance of the least squares solution?

Consider this tantalizing example. The two lines,
  \begin{equation*}   %  =   =   =   =   =
    \begin{split}
      y(x) &= \half - \half x, \\
      y(x) &= 1 - \half x,
    \end{split}
   %\label{eq:}
  \end{equation*}
are plotted in figure \ref{fig:tantalizing} along with the least squares solution is 
  \begin{equation*}   %  =   =   =   =   =
   %\begin{split}
      x_{LS} = \Ap b = \frac{1}{10}\mat{c}{ 3 \\ 6}.
   %\end{split}
   %\label{eq:}
  \end{equation*}
What is so special about this point?
\begin{figure}[htbp] %  figure placement: here, top, bottom, or page
   \centering
     \includegraphics[ width = 3in ]{\pathgraphics "tricks"/"parallel point"} 
   \caption{Parallel lines and the least squares solution.}
   \label{fig:tantalizing}
\end{figure}

\subsection{Intersecting Lines}
  \begin{equation*}   %  =   =   =   =   =
   \begin{split}
      y &= 1, \\
      y &= x .
   \end{split}
   %\label{eq:}
  \end{equation*}

  \begin{equation*}   %  =   =   =   =   =
   %\begin{split}
      \mat{rr}{0 & 1 \\ -1 & 1 }\mat{c}{x\\y} = \mat{c}{1\\0}
   %\end{split}
   %\label{eq:}
  \end{equation*}


\endinput